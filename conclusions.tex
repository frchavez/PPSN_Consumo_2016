\section{Conclusions}
\label{conclusions}

We have presented a preliminary study on the power consumption
behavior of a well known EA: the Genetic Programming algorithm.  We
have analyzed the behavior of a benchmark GP problem, the
multiplexer-6, and have run it using {\tt lilgp} on different
hardware devices running a number of operating systems, from blade
systems using different Linux distributions to tablet devices running
Android. 

One of the first thing we have learned is that although devices with
better processors can run the algorithm faster, they spend
larger amounts of energy, and the total energy required to find a
solution is also larger.  This means that although the standard
preference for better hardware platforms and processors allows to find
solutions more quickly, it incurs in waste of energy that should be
considered:  it is much more energy efficient to run the algorithm on
a Raspberry Pi or a tablet device. Of course, it is expected that some very 
computationally-expensive problems could not be fully solved in one
of these devices. Then again, they can provide a valuable, energy-efficient
contribution to the collective resolution of such problems when used
within a larger ephemeral network of computing devices \cite{self}.

Secondly, we have seen the influence of one of the main parameters
of the algorithm may have on the power consumed:  changing population
sizes automatically produce a change in the amount of energy required to reach
solutions, and this is a hint on future analysis.  We should thus carefully
consider how each of the parameters of the algorithm may also
influence the amount of energy consumed when looking for solutions. 

As a future line of work, it would be interesting to consider
different devices in the same class exploring the space of
solutions in a multiobjective way: which devices manage to find the
solution faster for the least amount of energy? We will expand studies
to other evolutionary algorithms to check whether these energy
profiles are exclusive of Genetic Programming or there are variations
among them. Energy profiling the algorithms will also allow us to find
out where the energy expenses actually come from, allowing us to
optimize the algorithm itself making it energy-aware.