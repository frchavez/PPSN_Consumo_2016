\section{Methodology}
\label{methodology}

This preliminary study tries to measure power consumption for 
%a couple of EAs, Genetic Algorithms (GA) and Genetic Programming (GP). 
the Genetic Programming (GP) algorithm. In the following subsections
both the algorithm setting (\ref{algsetting}) and computational
platforms (\ref{compplat}) are presented. 

\subsection{Algorithmic Setting}
\label{algsetting}

Given the stochastic nature of this kind of algorithms, we firstly
decided to run each of the experiment 30 times so that the average can
be computed as an estimation of the algorithm behavior. In order to
establish a fair comparison among the different hardware platforms
considered, these runs are done with the same 30 random seeds so that
all of the runs are exactly the same in every platform, when
considering high level operations defined in the high level
programming language. 

On the other hand, and given the influence of computing time in the
total amount of energy consumed by the algorithm, we configured the
main loop of the algorithm to finish when  
%a required level of solution quality is reached. 
the optimal solution is found.  We are thus mainly interested in the
average computing time for the 30 runs, together with the energy
consumed along that time.  The only differences that may arise are due
to hardware differences:  instruction set architecture, processor
speed and operating system;  % also the operating system
                               % architecture affects  -  pedro 
features that are not the focus of this work.  Nevertheless, these
differences may influence future decision on the preferred hardware
and operating system for the algorithms.

We must also mention the interest in studying some of the main
parameters of the algorithm:  they have a well-known impact on the
time to find solutions, and may thus also directly, or indirectly
influence the energy consumed to reach that solution.  In this
preliminary study we have focused on population size and have tested
several values for the problem selected.  Although we are working with
GP and a well-known problem, this first analysis will be helpful to
see that energy consumption is an important issue when working with
EAs. 

In order to ease the compilation processes in every hardware platform,
we have selected a well known implementation of GP in the C
programming language:  lilgp \cite{lilgp}. 
%on the one hand a simple GA \cite{michalewicz} with a simple
%C-implementation \cite{web_algoritmo}, and on the other hand lilgp
%for GP \cite{lilgp}. 
Regarding the problem selected for the experimental stage, we have
selected one of the test problems provided by lilgp: the multiplexer
problem. To be precise, we have set up to work with 6 bits. The main
parameters of the algorithm are described in Table~\ref{Table:par_gp}.
Function and terminal sets are the standard ones as described by Koza
\cite{koza}. 

%with each of the software tools selected:  the multiplexer problem provided with lilgp on the one hand, and an scalar function -check ~\ref{eq:fitness}-  with five variables to be optimized. 

%Main parameters for both the GA and the GP problems are described in Table~\ref{Table:par_ga} and 
%Main parameters are described in Table~\ref{Table:par_gp}.  %The upper limit for generation number is  $10000$, stopping the run when the best individuals is ($0,999\%$) near the optimal solution. Four population sizes have been tested.

\begin{table}[!t]
%\renewcommand{\arraystretch}{1.3}
\centering
\caption{Main GP Parameters for multiplexer 6.}
\label{Table:par_gp}
\begin{tabular}{cc}
\hline
%Max number of generation & $10000$ \\
Max number of generations & $500$ \\
%Population sizes & $50$,$100$,$500$,$1000$ \\
Population sizes & $100$, $200$, $400$, $500$, $1000$ \\
Crossover probability & $0.9$ \\ 
Mutation probability & $0.1$ \\ 
%Other parameters &  default values for the problem in \cite{lilgp} \\
\hline
\end{tabular}
\end{table}


\subsection{Computational Platforms}
\label{compplat}

Several computational platforms have been tested, i.e., \raspberrynsp, \tabletnsp, \laptopnsp, \iMac and \bladensp.  Table~\ref{Table:devices} provides the details for the hardware architectures and operating systems used. Given the differences among hardware devices, we have employed different ways  % means???  
for measuring energy consumed by each of the algorithms.  

\begin{table}[!t]
%\renewcommand{\arraystretch}{1.3}
\centering
\caption{Devices}
\label{Table:devices}
\begin{tabular}{lp{4.5cm}ccp{3cm}} \hline
Device		&	Processor			&	Cores	&	RAM &	OS		\\ 
\hline
\raspberry   	& Cortex-A7 900 MHz			&	 4 	&	1GB &	Raspbian GNU/Linux 7	\\
\tablet		& Samsung Galaxy Tab 3 SM-T311, Exynos 4212 1.5 GHz& 2  & 1.5 GB &	Android	4.4.2 (kernel 3.0.31) \\
\laptop 		& Intel(R) Core(TM) i5-2450M 2.5GHz	&	4	&	8GB &	Ubuntu 12.04.5 LTS \\
\iMac		& Intel(R) Core(TM) i5 2.7GHz	& 4	& 4GB 	&OSX 10.11.4			\\
\blade		& Virtual Machine (on IBM 8CPUs x 2GHz Intel Xeon CPUE5504 @ 2.00GHz (x2), 16Gb RAM) & 4 & 4GB & Debian 6.0, 64 bits\\
% this table should be updated with the information about the Android tablet:
%   SAMSUNG GALAXY TAB-3 (SM-T311)
%   Android 4.4.2
%   Kernel 3.0.31
%   CPU 1.5 GHz dual core
%   Samsung Exynos 4212 Dual SoC processor
%   1.5GB RAM
\hline
\end{tabular}
\end{table}


\subsubsection*{Laptop and Raspberry Pi.}
Regarding the \laptop and \raspberrynsp, we have employed a voltmeter for measuring total consumption of the device in two different scenarios:  (i) when the algorithm is not running and (ii) when the algorithm is running.  Given that \emph{current intensity} values remain constant in every case, we can obtain the measure for the algorithm by simply subtracting both values for a given device.  This value together with electric tension allows us to compute watts consumed.

\subsubsection*{Tablet.}
In order to collect data about energy consumption on Android devices, such as smartphones or tablets, \emph{PowerTutor}\footnote{\url{http://ziyang.eecs.umich.edu/projects/powertutor/documentation.html}} \cite{powertutor2} has been used. 
% http://ziyang.eecs.umich.edu/projects/powertutor/documentation.html
This app is a diagnostic tool for analyzing system and app energy consumption.
In order to obtain power measurements of the EA, PowerTutor runs in the background and logs data on power utilization for each app, summarizing the info in an intuitive user interface (reporting the number of Joules the app has consumed during the run).


\subsubsection*{iMac.}
Data collection in the iMac was done using \emph{HardwareMonitor}\footnote{\url{http://www.bresink.com/osx/HardwareMonitor.html}}. 
This application suite includes a command-line tool that provides readings of the internal
hardware sensors built on the computer. In order to obtain power measurements, a shell
script is run in parallel to each run of the EA. This script gathers sensor data periodically
(we use a sampling frequency of 1s) and goes to sleep state between measurements. 
During the experimentation, no other application is run, apart from background processes 
under OS control. 
To gauge the data,
the same data-collection process is run for 100s before each batch of runs, thus providing
an indication of the system basal consumption at that moment which is in turn used to 
compute the excess power consumption due to the EA (and hence accounting for
eventual hysteretic phenomena).

\subsubsection*{Blade.}

Ecosystems such as clusters are very used to compute big data sets. This kind of systems allow us to optimize, by sharing, resources like Ethernet, storage devices, power supply, etc. %In this work we have studied, for the problem described, if it is possible to optimize the energy consumption. We have used a virtualization ecosystem, where the problem has been executed in a virtual machine. 
The ecosystem we use employs VMWare Esxi 5.0\footnote{\url{https://www.vmware.com/support/vsphere5/doc/vsphere-esx-vcenter-server-50-release-notes.html}} (see Table \ref{Table:devices}) whose hypervisor provides us with information about energy consumed by both the hardware platform as well as any of the virtual machines running on it.  Thus we can obtain specific data for the virtual machine running the algorithm, and thus we can compute the difference between energy consumption when the algorithm is running and when it is not running.  
%, where there is a virtual machine with the features described in Table \ref{Table:devices}. The hypervisor system allows us to control both the energy consumed by the Blade and by the virtual machine in question. With these Blade features we can study the power measurements in the different problem configurations.

%\subsubsection*{FPGA.}


%Para realizar la experimentación, y tal como se describe en la sección previa, en primer lugar  hemos seleccionado el conjunto de dispositivos descritos en la Tabla~\ref{Table:devices}. 


%When the Xilinx FPGA has been tested using the evaluation kit described in Table~\ref{Table:devices} to compile the GA algorithm. Given the way the algorithms are converted into hardware components, the stop criteria cannot be applied, and instead, the average number of generations computed in other hardware platforms has been employed as the upper limit for the run, given that the same random seeds will produce the same high level operations for the algorithm.  The total ammount of generation for population sizes of 50, 100, 500 and 1000 individulas are  6505, 5859, 2732 y 1684, respectively.


%Las características particulares del entorno donde se implementa el algoritmo se especifican en la Tabla~\ref{Table:devices}.

%\begin{table}
%\renewcommand{\arraystretch}{1.3}
%\centering
%\caption{FPGA hardware}
%\label{Table:fpga}
%\begin{tabular}{cc} \hline
%  Version & Vivado v.2015.4 (win64) Build 1412921 Wed Nov 18 09:43:45 MST 2015 \\ 
%  Device & xc7z020clg484-1\\
%\hline
%\end{tabular}
%\end{table}

%When the FPGA SoC ZC702 has been employed, together with the IP module required for the algorithm, a proprietary Xilnx IP module was required to launch and connect the algorithm with input/output.  This means that a number of extra hardware resources are required (LUT, LUTRAM, FF, BRAM, DSP, BUFG) which will increase power consumption.  

%\begin{table}
%\renewcommand{\arraystretch}{1.3}
%\centering
%\caption{Característica de los dispositivos.}
%\label{Table:devices}
%\begin{tabular}{cccc} \hline
%Dispositivo&Procesador&Núcleos&RAM\\ \hline
%Raspberry Pi& Cortex-A7 (900 Mhz)& 4 &1GB \\
%Laptop & Intel(R) Core(TM) i5-2450M (2.50GHz)&4&8GB\\
%Xilinx ZYNQ-7000 SoC ZC702 (50Mhz) & & \\
%\hline
%\end{tabular}
%\end{table}

%\begin{equation}
%\label{eq:fitness}
% F=x_{1}^{4} + (x_{2}^{3} * x_{3}^{2}) - (x_{3} * x_{4}) + x_{5} 
%\end{equation}




%\begin{table}
%\renewcommand{\arraystretch}{1.3}
%\centering
%\tiny
%\caption{Main GA parameters.}
%\label{Table:par_ga}
%\begin{tabular}{cc}
%\hline
%Max number of generations & $10000$ \\
%Population sizes & $50$,$100$,$500$,$1000$ \\
%Crossover probability & $0.8$ \\ 
%Mutation probability & $0.15$ \\ 
%\hline
%\end{tabular}
%\end{table} 
