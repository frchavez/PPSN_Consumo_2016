\section{Evolutionary Algorithms and Power Consumption}
\label{eas}

Computer science took interest in energy efficiency a number of years
ago, and a new research topic was born, \textit{Green Computing}
\cite{green-computing}, together with the \textit{energy-aware}
\cite{energyaware,energy-aware} concept. Even processor 
makers offered new processors providing \textit{dynamic frequency
  scaling}, which adapts power consumption as well as heat dissipation
to the need of the processes to be run \cite{scaling,dynamic-scaling,energy-efficient}. 

On the other hand, EAs has already been applied as optimization
algorithms in the context of energy management.  We can thus find
optimization problems associated to HVAC (Energy management of
heating, ventilating and air-conditioning) \cite{HVAC},
\cite{chiller}.  We can also find EAs applied to energy dispatch
\cite{dispatch}.  But any of the above referred problems are only
tangentially connected to the problem we are interested in:  how to
include energy consumption as one of the main features of EAs to be
considered when looking for solutions, and its relationship with the
main parameters of the algorithm. 

The main concept discussed in this paper is the capability of an EA to adapt
to a dynamic environment, being power consumption one of the main
components to be optimized \cite{self}. This capability, which
is one of the self-$\star$ features of a
given algorithm, including EAs \cite{self}, has already been
considered by researchers in other kind of algorithms and computer architectures, in some cases
an essential part of them \cite{energy-aware}. Energy-awareness is
considered a key
component in infrastructures of any size, from large data centers to
processor architectures for mobile devices where battery life must be
optimized.  Also in this context, EAs have been employed to design
cache hierarchies reducing power consumption and heat dissipation
\cite{cache}. 

Yet, to the best of our knowledge, EAs have never been studied from
the point of view of their own energy needs. 
Given their stochastic nature, the number of parameters regulating the
way they perform the search process and the plethora of hardware
platforms available to run them, we consider it of interest to study
the energy consumed when looking for a solution, so that in the future
they may become energy-aware and capable of self-regulating when
progressing towards the solution of the problem faced. This is what we
have set out to do in this paper.

%This is part of our first approach to the study of EAs energy
%consumption.  Although we have already performed a preliminary study
%with EAs \cite{MAEB}, we try here to analyse a widespread software
%tool devoted to GP, \textit{lilgp}, so that we can reach wider
%conclusions. 

%I think that this paragrah can't be included, because the MEAB 2016 is later than PPSN. - Paco Chávez

% maybe this paragraph should be at the end of the introduction section instead of at the end of literature section  -  pedro
%The rest of the paper is organized as follows:  
% It's there already