\section{Introduction}

The analysis of evolutionary algorithms typically relies on computational complexity \cite{complexity}, %this is not an and. Change it - JJ
 and even when sometimes it is not possible to diminish it, processor
 companies try to offer new products providing higher performance.
 Thus, \textit{multi-core and many-core} processors allow to run
 parallel applications on desktop computers, but indirectly imply the
 need of learning new parallel programing paradigms and libraries. % don't see anything coming here - JJ  

Despite the advantages provided by these models, an important feature
is frequently forgotten:  power consumption, which largely correlates
with performances provided by new processors. That is why, in an
environment where raw processor speed is no longer doubling at an
accelerated pace, reducing power consumption and taking it into
account when evaluating algorithms becomes an issue, to the point that
latest HPC benchmarks also include this measurement in their reports
and there are calls for {\em energy-proportional computing}
\cite{barroso2007case} and 
% Thus, although new
% hardware capabilities allows to significantly reduce computing time,
% the cost of electricity required to run the algorithms is not always
% taken into account.% it is definitely not. When speed is not
                   % increasing, and it is not, the most important
                   % feature is consumption. You could talk about
                   % Moore's law here (or its end thereof) - JJ
                   
                   % Feel free to talk yourself, JJ.  - Paco.
% Done :-) - JJ
\textit{green computing} \cite{green-computing}, a term that was born in the last decade to refer
to problems associated to power consumption in computing environments,
particularly in large data centers. But this
energy-aware and proportional point of view is equally applicable to desktop computers
and any kind of algorithms that may be run. 

When dealing with evolutionary algorithms (EAs), big efforts have been
applied to improve performances while applying parallel and
distributed systems \cite{paba}.  Improvements have tried to analyse
global quality of solutions when compared with time required to find
them.   
But similarly as the traveller considering not only speed but also
price when selecting means of transport, we should also consider
energy consumption when running an algorithm, and not just the time to
solution.  % this sentence (or similar) should be included in the
           % abstract  -  pedro 

To the best of our knowledge, the influence of this important
parameter has not been analysed yet in the context of EAs, although
its importance has already been recognised \cite{ephemeral}.  This is
the main goal of this work, to make a preliminary analysis of the
impact of power consumption when running EAs on different hardawre
architectures, so that we may in the future be aware of the
importance, and even design energy-aware EAs. 

The rest of the paper is organized as follows:  Section \ref{eas}
describe previous works on the area;  section \ref{methodology}
describes the experiments performed and section \ref{results} shows
the results obtained.  Finally we summarize our conclusions in section
\ref{conclusions}. 

% I think we should include here the description of the paper structure, i.e., the "The rest of the paper is structured as follows:"
%  - pedro

